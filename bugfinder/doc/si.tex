\documentclass[12pt]{article}
\usepackage[margin=2.7cm]{geometry}
\usepackage{amsmath}
\usepackage{amssymb}
\begin{document}
\title{Notes on the calculation of the correlated mixed expectation
values}
\author{
Kevin E. Schmidt\\
Department of Physics\\
Arizona State University\\
Tempe, AZ 85287 USA\\
}
\maketitle
\section{Notation}
\begin{itemize}
\item
Label the orbital states as $|k\rangle$ with a $k$ index.
\item
The position and spin state
of particle $i$ is written $|\vec r_i$ $s_i\rangle$.
\item
We store a particle
spin state as the 4 numbers
$\langle p\uparrow|s_i\rangle$,
$\langle p\downarrow|s_i\rangle$,
$\langle n\uparrow|s_i\rangle$,
$\langle n\downarrow|s_i\rangle$, which is written as
$\langle s|s_i\rangle$ with $s$ running from 1 to 4 respectively.
\item
We use the matrix identity 
${\rm det} S^{-1}S' = \frac{{\rm det} S'}{{\rm det}S}$ to calculate
the determinant of the matrix $S'$ when it has only a small number
of columns diffrent from $A$. The unchanged columns give $1$ on the
diagonal and $0$ on the off diagonals so that the determinant of $C$
with $C=S^{-1}S'$ is given by the determinant of the determinant
of the matrix $C_{mn}$ where $m$ and $n$ only take values of the changed
columns of $S'$.
\end{itemize}

\section{Calculations}
Here we deal with the case where the operators only change the spin state
of the particles. The Slater matrix is
\begin{equation}
S_{ki} = \langle k|\vec r_i s_i\rangle =
\sum_{s=1}^4\langle k|\vec r_i s\rangle \langle s|s_i\rangle
\end{equation}
so a general matrix element of $S'$ will be a linear combination of
the orbital matrix elements
$\langle k|\vec r_i s\rangle$. We therefore precompute
\begin{eqnarray}
{\rm sxmallz(j,s,i)} &=& \sum_k S^{-1}_{jk} \langle k|\vec r_i, s\rangle
\end{eqnarray}
where $s$ runs from 1 to 4. $j$ and $i$ run from 1 to the number of
particles $A$. This quantity is then used for all subsequent determinant
calculations. The original reason for having the spin as the middle index
was to exploit the matrix multiply form.
However, we can get a useful speed up by having the spin
variable leftmost so that it has unit stride in inner loops, so
we transpose this and write
\begin{eqnarray}
{\rm sxz(s,i,j)} &=&
{\rm sxmallz(j,s,i)} = \sum_k S^{-1}_{jk} \langle k|\vec r_i, s\rangle
\end{eqnarray}

\section{Matrix elements of 1-, 2-, and 3-body spin-isospin operators}
We need mixed expectations of various 1-, 2-, and 3-body spin-isospin
operators to calculate the various terms in the local energy. Since
the correlated trial wave function will be a linear combination of
many terms, it is most efficient to organize the calculations so that
the fewest operations are done in the inner loops.

\subsection{1-body spin-isospin operators}
We often would like to have the mixed expectation values of
the 15 operators given by the components of $\sum_i \vec \sigma_i$,
$\sum_i \vec \tau_i $,
$\sum_i \vec \sigma_i \vec \tau_i $. Rather than calculate their
matrix elements in the inner loops, it is more efficient to write
a general 1-body spin/isospin operator operating on a walker
$|s_1,s_2,...,s_A\rangle$,
as the linear combination of the $4A$ terms
\begin{equation}
\sum_i O_i |s_1,...,s_A\rangle =
\sum_{s=1}^4 \alpha_{1s}|s,s_2,...,s_A\rangle
+\sum_{s=1}^4 \alpha_{2s}|s_1,s,...,s_A\rangle+...
\end{equation}
where $\alpha_{is} = \langle s|O_i|s_i\rangle$. The code then
calculates the intermediate quantity given by the matrix elements
\begin{equation}
{\rm d1b(s,i)} =
\frac{\langle \Phi|R,s_1,...,s_{i-1},s,s_{i+1},...,s_A\rangle}
{\langle \Phi|R,s_1,...,s_A\rangle}
\,,
\end{equation}
where d1b standards for ``distribution 1-body'' since this looks like
a one-body spin projector distribution.

Given the sxz(s,i,j) corresponding to $|\Phi\rangle$, we have
\begin{equation}
{\rm d1b(s,i)} = {\rm sxz(s,i,i)} \,.
\end{equation}

The full trial function will be expanded below in the equivalent of
a linear combination of many different $\langle \Phi|$ pieces. These
then give a full d1b(s,i) from the same linear combination of
the separate ones.

Once the full d1b is calculated, the desired one-body operator mixed
expectations are calculated from
\begin{equation}
\sum_{i=1}^A\sum_{s=1}^4 \alpha_{is}{\rm d1b(s,i)} \,.
\end{equation}

\subsection{2-body spin-isospin operators}
We now repeat the above analysis for 2-body operators. Since the pair
1-2 is equivalent to 2-1, we write the 2-body distributions we need
given a particular $\langle \Phi|$
as
\begin{equation}
{\rm d2b(s,s',ij)} =
\frac{\langle \Phi|R,s_1,...,s_{i-1},s,s_{i+1}
,...,s_{j-1},s',s_{j+1},...s_A\rangle}
{\langle \Phi|R,s_1,...,s_A\rangle}
\,,
\end{equation}
where $i<j$, and $ij$ runs from 1 to $A(A-1)/2$. The d2b contribution
is calculated from the sxz as the corresponding
$2\times 2$ subdeterminant
\begin{equation}
\rm {d2b(s,s',ij)} = \rm {det}
\left (
\begin{array}{cc}
\rm {sxz(s,i,i)} & \rm {sxz(s,i,j)}\\
\rm {sxz(s',j,i)} & \rm {sxz(s',j,j)}\\
\end{array}
\right ) \,.
\end{equation}

Just as for the 1-body case, there will a linear combination of
such terms to give the full d2b result. Once this is calculated, the
mixed expectation contribution of any
two-body spin-isospin operator will be given by
\begin{equation}
\sum_{s=1}^4 \sum_{s'=1}^4 \sum_{ij=1}^{((A-1)A)/2} {\rm d2b(s,s',ij)}
\langle s s'|O_{ij}|s_i s_j\rangle
\end{equation}

\subsection{3- or higher-body contributions}
The generalization to higher-body operators should now be obvious.
For example, for 3-body operators, we define a d3b(s,s',s'',ijk)
where $s$, $s'$, and $s''$ are substitued for $s_i$, $s_j$, and $s_k$,
$i<j<k$,
and all $4^3$ spin-isospin terms are calculated.
The result will be the corresponding subdeterminants
\begin{equation}
{\rm d3b(s,s',s'',ijk)} = \rm {det}
\left (
\begin{array}{ccc}
\rm {sxz(s,i,i)} & \rm {sxz(s,i,j)} & \rm {sxz(s,i,k)} \\
\rm {sxz(s',j,i)} & \rm {sxz(s',j,j)} & \rm {sxz(s',j,k)} \\
\rm {sxz(s'',k,i)} & \rm {sxz(s'',k,j)} & \rm {sxz(s'',k,k)} \\
\end{array}
\right )\,,
\end{equation}
and the contributions to the three-body mixed expectations are
\begin{equation}
\sum_{s=1}^4 \sum_{s'=1}^4 \sum_{s''=1}^4
\sum_{ijk=1}^{((A-2)(A-1)A)/6} {\rm d3b(s,s',s'',ijk)}
\langle s s' s''|O_{ijk}|s_i s_j s_k\rangle \,.
\end{equation}

I have seen some references to other calculations where physical quantities
require the calculations of all the $m\times m$ subdeterminants
of an $N \times N$ matrix. I have not found any mention of an
optimal algorithm. Keep your eyes open for this since these are currently
some of the most expensive parts of the code.

\section{The correlated wave function}
Initially a sum of pair operator correlations was used, and we used
pair updates to the inverse matrix in sxz since we worried that
using a sequence of single operator updates might cause a division
by zero or loss of precision from division by a small number,
if the determinant of the matrix was zero for an unphysical
intermediate state. This can easily happen if, for example, all the spinors
have components of a single charge. An intermediate state with an $s$ value
having a flipped charge
would be orthogonal to the model state and would give
a zero determinant, even though flipping a second charge in the second
operator of a pair would give a well defined result.

However, for realistic walkers, we have found that this division by
zero does not occur. Therefore, we now use a sequence of single operator
updates to obtain 1-, 2-, or more-particle additive correlations.
This greatly simplifies the code, and makes it easy to add additional
operators.

In order to use the same single determinant 
code described above to calculate the mixed expectations,
we need the equivalent of
${\rm sxz(s,m,n)}$ for the ``Slater matrix'' (note this is no longer
antisymmetric under interchange of particles since we only operate
on particle i)
\begin{equation}
S'_{km} = \left \{
\begin{array}{cc}
S_{km} & m \neq i\\
\langle k|O_i|\vec r_i s_i\rangle & m = i
\end{array}
\right . \,.
\end{equation}
We require the inverse of this matrix $S'^{-1}$ multiplied by the
possible ``new orbitals'' which will be
\begin{equation}
\label{eq6}
S'_{km}(s)= \left \{
\begin{array}{cc}
\langle k|\vec r_m s\rangle & m\neq i\\
\langle k|O_i|\vec r_i s\rangle &  m=i\\
\end{array}
\right . \,.
\end{equation}
\begin{equation}
{\rm sxzi(s,n,m)}=\sum_k S'^{-1}_{mk} S'_{kn}(s) \,.
\end{equation}

We can calculate the updated inverse in the usual way by calculating the
ratio of determinants when two columns are changed to the original
determinant in two ways where we first change column $i$ to give $S'$,
and then column $j$ to give $S''$,
\begin{equation}
\frac{{\rm det} S''}{{\rm det} S} = \frac{{\rm \det}S'}{{\rm det}S}
\sum_n S'^{-1}_{jn}S''_{nj} =
\left \{
\begin{array}{cc}
\sum_{nm} S^{-1}_{im} S''_{mi} S^{-1}_{jn} S''_{nj}
-\sum_{nm} S^{-1}_{in} S''_{ni} S^{-1}_{jn} S''_{ni} & j \ne i\\
\sum_n S^{-1}_{in} S''_{nj} & j = i\\
\end{array}
\right . \,.
\end{equation}
Realizing that $S''_{mi} = S'_{mi}$ when $j \ne i$, we equate
the terms multiplying $S''_{nj}$ to find usual result
\begin{equation}
S'^{-1}_{jn} = \left \{
\begin{array}{cc}
S^{-1}_{jn} -\frac{\sum_m S^{-1}_{jm}S'_{mi}}{\sum_\ell S^{-1}_{i\ell}
S'_{\ell i}} S^{-1}_{in}  & j \ne i\\
\frac{S^{-1}_{in}}{\sum_m S^{-1}_{im}S'_{mi}} & j = i\\
\end{array}
\right . \,.
\end{equation}

Multiplying we find
\begin{equation}
{\rm sxzi(s,n,m)} =  \left \{
\begin{array}{cc}
{\rm sxz(s,n,m)} -
\frac{\sum_k S^{-1}_{mk}\langle k |O_i|\vec r_i s_i\rangle}
{\sum_k S^{-1}_{ik}\langle k |O_i|\vec r_i s_i\rangle} {\rm sxz(s,n,i)} & n,m \ne i\\
\frac{\rm sxz(s,n,i)}
{\sum_k S^{-1}_{ik}\langle k |O_i|\vec r_i s_i\rangle} & m=i; n \ne i\\
\sum_k S^{-1}_{mk}\langle k |O_i|\vec r_i s\rangle
-
\frac{\sum_k S^{-1}_{mk}\langle k |O_i|\vec r_i s_i\rangle}
{\sum_k S^{-1}_{ik}\langle k |O_i|\vec r_i s_i\rangle}
\sum_k S^{-1}_{ik}\langle k |O_i|\vec r_i s\rangle
& n=i; m \ne i\\
\frac{\sum_k S^{-1}_{ik} \langle k|O_i|\vec r_i s\rangle}
{\sum_k S^{-1}_{ik} \langle k|O_i|\vec r_i s_i\rangle} & n=m=i\\
\end{array}
\right . \,.
\end{equation}
Writing the sums in terms of sxz(s,n,m), we define
\begin{eqnarray}
{\rm opi(s,m)} =
\sum_k S^{-1}_{mk}\langle k |O_i|\vec r_i s\rangle
= 
\sum_{k,s'} S^{-1}_{mk}\langle k | \vec r_i s'\rangle \langle s'|O_i|s\rangle
=  \sum_{s'} {\rm sxz(s',i,m)} \langle s'|O_i|s\rangle \,.
\end{eqnarray}
The name opi is for ``operator i.''
In the code, we have the function opmult that operates with the
15 possible cartesian 
1-body spin-isospin operators (3 components of $\vec \sigma$, 3
components of $\vec \tau$, and 9 components fo $\vec \sigma \vec \tau$)
to the right onto $|s\rangle$.
We use this to operate to the left by taking the adjoint, operating
and then taking the adjoint again. This amounts to taking the
complex conjugate of sxz, applying the operator, and taking its complex
conjugate again. Alternatively, the code could be changed to use the
complex conjugate of the operator. Either one of these 15 operators if we
are using cartesian components, or a linear combination of them defines
opi(s,m). The other sum we need is where the right hand spinor is $|s_i\rangle$.
We define
\begin{equation}
{\rm di(m)} = 
\sum_k S^{-1}_{mk}\langle k |O_i|\vec r_i s_i\rangle
= \sum_s {\rm opi(s,m)} \langle s|s_i\rangle
\end{equation}
where $\langle s|s_i\rangle$ are the walker spinor components sp(s,i).
Since this is in some sense a ratio of determinants, the name di
stands for ``determinant i.''
Rewriting sxzi in terms of sxz, opi, and di, we have
\begin{equation}
{\rm sxzi(s,n,m)} =  \left \{
\begin{array}{cc}
{\rm sxz(s,n,m)} -
\frac{{\rm di(m)}}
{{\rm di(i)}} {\rm sxz(s,n,i)} & n,m \ne i\\
\frac{\rm sxz(s,n,i)}
{{\rm di(i)}} & m=i; n \ne i\\
{\rm opi(s,m)} - \frac{{\rm di(m)}}{\rm di(i)} {\rm opi(s,i)} & n=i; m \ne i\\
\frac{\rm opi(s,i)}{\rm di(i)} & n=m=i\\
\end{array}
\right . \,.
\end{equation}

Handing this new sxzi to the mixed expectation value routines will
give the expectations with the modified $\langle \Phi|$ in the denominator.
Therefore these results are multiplied by the ratio of the new to old
determinants in order to be added to the d1b, d2b, and d3b sums.

At this point, 2-body, 3-body, etc. operators can be done by repeating
the calculation above starting with sxzi and repeating the above update.

\section{Calculation of just the wave function}
If only the wave function for a new walker is needed, the code uses
the machinery developed above to calculate d1b, d2b, d3b for just
the single Slater determinant. The operator correlations are
converted into the $p\uparrow$, $p\downarrow$, $n\uparrow$, $n\downarrow$
basis by operating on the walker spinors with the opmult(sp) routine,
and then combining these with the operator correlations. For example
a correlation containig $f_{\alpha \beta}(r_{ij})
\sigma_{i\alpha}\sigma_{j\beta}$
is included by f2b(iz,jz,ij)=f2b(iz,jz,ij)+$f_{\alpha\beta}(r_{ij})$*
spx(iz,$\alpha$,i)*spx(jz,$\beta$,j). The determinant ratio is
then given by sum(d2b*f2b) and similarly for 1- and 3-body correlations.

\section{Breaking up the correlations}
When operator mixed expectation values are needed, we will require
the calculation of the d1b, d2b, d3b, etc. for every new sxz. Since
these calculations will appear in the inner loop, we would like to
break up the operator correlations in such a way to minimize their
calls. On the other hand, updates of sxz require order $A^2$ operations.
So we also want to minimize these updates.

The most straightforward
break up of the $v_6$ correlations uses the cartesian components of
the pauli spin-isospin operators. There are then 39 operator pairs
for each particle, (3 from $\tau_\alpha \tau'_\alpha$, 9 from
$\sigma_\alpha \sigma_\beta'$, and 27 from $\sigma_\alpha\sigma_\beta'
\tau_\gamma\tau_\gamma'$). However, for any particle $j$, particle
$i$ can be operated on by just 15 operators $3 \sigma_\alpha$,
3 $\tau_\alpha$, and 9 $\sigma_\alpha\tau_\beta$. Therefore, we can loop
over the pair correlation with $i$ the outer loop. For a given $i$
value, we can calculate the 15 possible updates that give sxzi.
In the inner $j$ loop we then have 39 updates which each start from
a precomputed sxzi. After the $j$ updates, the d1b, d2b, d3b etc. can
be calculated. The inner loop will therefore have roughly $39A^2/2$
updates, each costing $A^2$ operations itself, along with order
$A$ operations for d1b, $A^2$ operations for d2b, and $A^3$ operations for d3b.

An alternative method reduces the number of operator pairs by
diagonalizing (or using singular value decomposition when the matrices
are not symmetric) the coupling terms in the $v_6$ operators. For example
the $\vec \sigma_i \cdot \vec \sigma_j$ and tensor components of the interaction
can be written as
\begin{equation}
\sum_{\alpha \beta} \sigma_{i\alpha} A^{(ij)}_{\alpha \beta} \sigma_{j\beta}
\,.
\end{equation}
Finding the eigenvectors $\psi^{(n)}_\alpha$ and eigenvalues $\lambda_n$
of the real symmetric matrix $A$, we can write this as
\begin{equation}
\sum_{\alpha \beta} \sigma_{i\alpha} A^{(ij)}_{\alpha \beta} \sigma_{j\beta}
= \sum_n \left [\sum_\alpha \sigma_{i\alpha} \psi^{(n)}_\alpha \right]
\left [\sum_\beta\sigma_{j\beta} \psi^{(n)}_\beta \right]
\,.
\end{equation}
and therefore define $O_i = \sum_\alpha \sigma_{i\alpha}\psi^{(n)}_\alpha$
$O_j = \sum_\beta \sigma_{j\beta}\psi^{(n)}_\beta$.
In this way we can reduce the number of operator pairs from 39 to 15.
There is a trade off however. Unlike the cartesian case, the $O_i$ are
different for each pair. Therefore both updates (unless
someone can find a clever way to combine cartesian updates to get these
updates) must be done in the j loop. We therefore reduce the
number of updates from 39 to 30. We reduce the number of calls to
d1b, d2b and d3b from 39 to 15.

For the case of 3-body anticommutator additive correlations, the
$A$ matrix above is no longer symmetric. However, this can be
handled using singular value decomposition where we write
\begin{equation}
A_{\alpha\beta} = \sum_\gamma U_{\alpha\gamma} \Lambda_\gamma V_{\gamma \beta}
\end{equation}
and everything above still goes through.

\end{document}

