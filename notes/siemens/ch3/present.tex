\documentclass{beamer}
\usepackage{amsmath}
%\usepackage{beamerthemesplit} % new 
\usetheme{Madrid}
\usefonttheme[onlymath]{serif}
\setbeamertemplate{frametitle}[default][center] %center slide titles

\begin{document}
\title{Notes on Siemens Ch. 3}
\author{Cody Petrie} 
\date{\today} 

%start slides
\frame{\titlepage} 

\frame{\frametitle{The Black Sphere}
\begin{itemize}
   \item Model neutron scattering from nuclei as a particle being absorbed by spherical object.
   \item Start by expanding an incident plane wave in terms of spherical harmonics.
   \begin{align}
      e^{i\mathbf{k}\cdot\mathbf{r}} &= \sum\limits_l C_l Y_l^0(\theta) \\
      e^{ikz} &\approx \sum\limits_l \frac{\sqrt{\pi}}{kr} \sqrt{2l+1} \, i^{l+1} \left( e^{-i(kr-\frac{l\pi}{2})} - e^{i(kr-\frac{l\pi}{2})} \right) Y_l^0(\theta)
   \end{align}
   \item Here we have used the fact that $\mathbf{k}\cdot\mathbf{r}$ only depends on $\theta$, and not on $\phi$, thus $m=0$. Also, we have used various identities and the orthonormality of spherical harmonics.
\end{itemize}
}

\frame{\frametitle{The Black Sphere}
\begin{itemize}
   \item Scattering only happens for short time
   \begin{equation}
      \phi(r\rightarrow\infty) = \sum\limits_l \frac{\sqrt{\pi}}{kr} \sqrt{2l+1} \, i^{l+1} \left( e^{-i(kr-\frac{l\pi}{2})} - \eta_l e^{i(kr-\frac{l\pi}{2})} \right) Y_l^0(\theta)
   \end{equation}
   \item Scattered wave is just the total wave function minus the incident wave function, $\phi_{sct} = e^{ikz}-\phi(r\rightarrow\infty)$.
   \begin{equation}
      \phi(r\rightarrow\infty) = e^{ikz} + f(\theta)\frac{e^{ikr}}{r}
   \end{equation}
   \begin{equation}
      f(\theta) = \sum\limits_l i \frac{\sqrt{\pi}}{k}\sqrt{2l+1} Y_l^0(\theta) (1-\eta_l)
   \end{equation}
\end{itemize}
}

\end{document}
