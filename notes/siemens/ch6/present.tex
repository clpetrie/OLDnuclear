\documentclass{beamer}
\usepackage{amsmath}
%\usepackage{beamerthemesplit} % new 
\usetheme{Madrid}
\usefonttheme[onlymath]{serif}
\setbeamertemplate{frametitle}[default][center] %center slide titles

%My commands
\newcommand{\ket}[1]{\left| #1 \right>}
\newcommand{\bra}[1]{\left< #1 \right|}
\newcommand{\braket}[2]{\left<\left. #1 \right| #2 \right>}

\begin{document}
\title{Notes on Siemens Ch. 6}
\author{Cody Petrie} 
\date{\today} 

%start slides
\frame{\titlepage} 

\frame{\frametitle{Interactions Beyond the Mean Field}
\begin{itemize}
   \item The mean field approximation gives us basic features of nuclei. But now we're going to move beyond the mean field approximation.
   \item The first thing we are going to do is look at the pairing term to the binding energy (equation 4.3.2).
   \begin{equation}
      B_p = \frac{\left[(-1)^N+(-1)^Z\right]\delta}{A^{1/2}}
   \end{equation}
   \item This gives even-even nuclei a tighter binding energy. Also it turns out that the ground state of even-even nuclei have zero angular momentum.
   \item To explain these things we are going to go beyond the independent-particle motion (mean field).
\end{itemize}
}

\frame{\frametitle{Interactions Beyond the Mean Field}
\begin{itemize}
   \item  Add a perturbation to the mean field Hamiltonian ($H_R$ is called the residual interaction)
   \begin{equation}
      H = H_{MF} + H_R
   \end{equation}
   \begin{itemize}
      \item The eigenstates of $H_{MF}$ are Slater determinants (Uncorrelated SD's?).
   \end{itemize}
   \item One solution to this is to diagonalize $H_R$ in the $H_{MF}$ basis, but this requires large calculations.
   \item We are going to use other methods in this chapter. We will split (crudely) into long-range and short-range parts, and look at short-range parts here.
\end{itemize}
}

\frame{\frametitle{The $\delta$-Force}
\begin{itemize}
   \item Look at degenerate states of $H_{MF}$ because $H_R$ with have a decisive influence.
   \item Start with $H_{MF}$ in a full $j$ state and two identical nucleons in the next $j$ state.
   \begin{align}
      \psi^{nlj}_{JM}(1,2) &= \sum\limits_{m_1m_2} \braket{jm_1jm_2}{JM} \mathcal{A}\left[\Phi_{nljm_1}(1)\Phi_{nljm_2}(2)\right] \\
      \Phi_{nljm_1} &= \frac{1}{2} u_{nlj}(r) \sum\limits_{m,s} \braket{lm\frac{1}{2}s}{jm_1}Y_l^m(\theta,\phi)\chi_s
   \end{align}
   \item The shortest range for $H_R$ is a $\delta$-force.
   \begin{equation}
      H_R = V_0 \delta(\mathbf{r}_1 - \mathbf{r}_2)
   \end{equation}
\end{itemize}
}

\frame{\frametitle{The $\delta$-Force}
\begin{itemize}
   \item This Hamiltonian gives an energy
   \begin{align}
      E_R &= V_0 \int \psi^*_{JM}\delta(\mathbf{r}_1-\mathbf{r}_2)\psi_{JM}d^3\mathbf{r}_1 d^3\mathbf{r}_2 \\
      &= \frac{V_0\left[1+(-1)^J\right](2j+1)^2}{32\pi (2J+1)}\left|\braket{j,\frac{1}{2},j,-\frac{1}{2}}{J,0}\right|^2 \int\limits_0^\infty r^{-2}u^4_{nlj(r) dr}
   \end{align}
   \item Note here that $E_R$ vanishes for odd values of $J$. This means that two identical Fermi particles in the same j-shell can only be in even angular-momentum states.
   \item For an attractive force ($V_0<0$) the lowest energy has $J=0$ and the first excited state is $J=2$.
\end{itemize}
}

\frame{\frametitle{The $\delta$-Force}
\begin{itemize}
   \item For all $j> \frac{3}{2}$ the difference in energies of these two states is
   \begin{equation}
      \left|(E_2-E_0)/E_0\right| \approx \frac{3}{4}
   \end{equation}
   which is large as seen in figure 6.2 of the book.
   \item The two nucleons have their largest spatial overlap in this state ($J=0$).
   \item Thus an attractive $\delta$-interaction decreases the energy.
\end{itemize}
}

\frame{\frametitle{The Degenerate Pairing Model}
\begin{itemize}
   \item A main feature of the $\delta$-force that is maintained in the pairing force is that it only has non-zero matrix elements between time-reversed states. Also, they non-zero elements are all identital
   \begin{equation}
      \bra{jm_1\overline{jm_1}}V\ket{jm_2\overline{jm_2}} \equiv -G
   \end{equation}
   \item Let's use the basis states $j+\frac{1}{2} \equiv \Omega$. Now the Schr\"odinger equation becomes
   \begin{equation}
      -G\begin{pmatrix} 1 & 1 & \cdots & 1 \\ 1 & \hfill & \hfill & 1 \\ \vdots & \ddots & \vdots & \hfill \\ 1 & \hfill & \cdots & 1\end{pmatrix}
      \begin{pmatrix} x_1 \\ x_2 \\ \vdots \\ x_\Omega \end{pmatrix}
      = E\begin{pmatrix} x_1 \\ x_2 \\ \vdots \\ x_\Omega \end{pmatrix}
   \end{equation}
   \begin{equation}
      -G(x_1 + \cdots + x_\Omega) = Ex_1 = Ex_2 = \cdots = Ex_\Omega
   \end{equation}
\end{itemize}
}

\frame{\frametitle{The Degenerate Pairing Model}
\begin{equation}
   -G(x_1 + \cdots + x_\Omega) = Ex_1 = Ex_2 = \cdots = Ex_\Omega
\end{equation}
\begin{itemize}
   \item This has solutions
   \begin{equation}
      E = -G\Omega, \qquad \vec{x} = \frac{1}{\sqrt{\Omega}}(1,1,\cdots,1)
      \label{eq:groundenergy}
   \end{equation}
   and
   \begin{equation}
      E=0, \qquad x_1+x_2+\cdots+x_\Omega = 0.
      \label{eq:excitedenergy}
   \end{equation}
   \item Equation \ref{eq:groundenergy} refers to the $J=0$ state and the degerarate $J>0$ states have energy of equation \ref{eq:excitedenergy}.

\end{itemize}
}

\frame{\frametitle{The Degenerate Pairing Model}
\begin{itemize}
   \item Now assume we have $n$ particles in the j-shell ($n\le2\Omega$), and $p$ pairs of particles, i.e. $J=0$ states.
   \item Using equation \ref{eq:groundenergy} and the fact that $\Omega$ is the number of possible pairs we get
   \begin{equation}
      E(n,p)=-Gp\left(\Omega-n+p+1\right)
   \end{equation}
   \item Introduce \textit{seniority}, $S=n-2p$, the number of unpaired nucleons.
   \begin{align}
      E(n,S) &= -\frac{G}{4}(n-S)(2\Omega-n-S+2) \\
      E(n,0) &= -\frac{1}{4}Gn(2\Omega-n+2) \\
      E(2,0) &= -G\Omega \\
      E(2\Omega,0) &= -Gn/2
   \end{align}
\end{itemize}
}

\frame{\frametitle{The Degenerate Pairing Model}
\begin{itemize}
   \item The pairing force creates as many pairs of particles as possible.
   \item Even-even have zero spin.
   \item Odd numbers of nuclei, spin is determined by unpaired nucleon.
   \item Comparing odd-mass and even-mass nuclei we get,
   \begin{equation}
      E(2p+1,p)-E(2p,p) = Gp
   \end{equation}
   which seems to agree with experiment.
   \item For even-even nuclei, the lowest excited state is that of one broken pair, which has symmetric energy
   \begin{equation}
      E(2p,p-1)-E(2p,p)=G\Omega
   \end{equation}
\end{itemize}
}

\end{document}
