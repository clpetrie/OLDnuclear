\section*{Auxiliary Field Diffusion Monte Carlo}
The methods described above are adequate for describing spin-independent potentials, but as described in \cite{schmidt1999} the number of spin-isospin states is 
\begin{equation}
  \frac{A!}{Z!(A-Z)!}2^A,
\end{equation}
where $A$ is the nucleons and $Z$ is the number of protons, which is exponential in the number of particles. The number of states to be sampled quickly becomes large. The AFDMC method is then used to simplify these methods.

My goal is to improve the trial function used in the current version of the code and so my comments will be directed toward this goal. The simplest trial wave function you can have is a slater determinant, which is the determinant of all of the single particle orbitals.
\begin{equation}
  \Psi_T(\mathbf{r}) \doteq \left| \\
  \begin{array}{cccc}
    \phi_1(\mathbf{r_1}) & \phi_2(\mathbf{r_1})  & \cdots & \phi_N(\mathbf{r_1})  \\
    \phi_1(\mathbf{r_2})  & \phi_2(\mathbf{r_2})  & \cdots & \phi_N(\mathbf{r_2})  \\
    \vdots & \vdots & \ddots & \vdots \\
    \phi_1(\mathbf{r_N})  & \phi_2(\mathbf{r_N})  & \cdots & \phi_N(\mathbf{r_N})  
  \end{array} \right|
\end{equation} 
This trial wave function does not include correlations. Ideally correlations would be included as
\begin{equation}
  \left<\mathrm{RS}\middle|\Psi_T\right> = \left<\mathrm{RS}\right| \prod\limits_{i<j}\left[f_c(r_{ij})\left[1+\sum\limits_{i<j,p}u_{ij}^p\mathcal{O}_{ij}^p\right]\right] \left|\Phi\right>,
  \label{equ:corrfull}
\end{equation}
where the sum on $p$ in our code goes from $1$ to $6$, where the first 6 $\mathcal{O}_{ij}^p$ are $1$, $\mathbf{\sigma}_i \cdot \mathbf{\sigma}_j$, $3\mathbf{\sigma}_i \cdot \hat{r}_{ij}\mathbf{\sigma}_j \cdot \hat{r}_{ij} - \mathbf{\sigma}_i \cdot \mathbf{\sigma}_j$, and the same things multiplied by $\mathbf{\tau}_i \cdot \mathbf{\tau}_j$. However, this requires too many operations. As discussed in \cite{gandolfi2014} the correlations have currently been included as
\begin{equation}
  \left<\mathrm{RS}\middle|\Psi_T\right> = \left<\mathrm{RS}\right| \left[\prod\limits_{i<j}f_c(r_{ij})\right]\left[1+\sum\limits_{i<j,p}u_{ij}^p\mathcal{O}_{ij}^p\right] \left|\Phi\right>.
  \label{equ:corrsimp}
\end{equation}
This is a first approximation and does not include many terms in the interaction. The next step to move from equation~\ref{equ:corrsimp} to equation~\ref{equ:corrfull} is to add the independent pair terms. This would look like
\begin{equation}
  \left<\mathrm{RS}\middle|\Psi_T\right> = \left<\mathrm{RS}\right| \left[\prod\limits_{i<j}f_c(r_{ij})\right]\left[1+\sum\limits_{i<j,p}u_{ij}^p\mathcal{O}_{ij}^p + \sum\limits_{i<j,p}\sum\limits_{k<l,p}u_{ij}^p\mathcal{O}_{ij}^pu_{kl}^p\mathcal{O}_{kl}^p\right] \left|\Phi\right>. 
\end{equation}
I am currently working on understanding the code so that I can add these extra terms into the trial wave function.

Even though my project does not deal with the sampling of the spin states I will mention the idea here for completeness. As discussed in \cite{schmidt1999} the non-central part of the potential can be written in the form
\begin{equation}
  V_\mathrm{nc} = \frac{1}{2} \sum\limits_{n=1}^{3N}(O_n^{(\sigma)})^2\lambda_n^{(\sigma)} + \frac{1}{2} \sum\limits_{\alpha=1}^{3} \sum\limits_{n=1}^{3N}(O_{n\alpha}^{(\sigma\tau)})^2\lambda_n^{(\sigma\tau)} + \frac{1}{2} \sum\limits_{\alpha=1}^{3} \sum\limits_{n=1}^{N}(O_{n\alpha}^{(\tau)})^2\lambda_n^{(\tau)},
\end{equation}
where the three squared operators are given by
\begin{equation}
  \begin{split}
    O_n^{(\sigma)} &= \sum\limits_i \boldsymbol{\sigma}_i \cdot \boldsymbol{\psi}_n^{(\tau)}(i) \\
    O_{n\alpha}^{(\sigma\tau)} &= \sum\limits_i \tau_{i\alpha} \boldsymbol{\sigma}_i \cdot \boldsymbol{\psi}_n^{(\sigma\tau)}(i) \\
    O_{n\alpha}^{(\tau)} &= \sum\limits_i \tau_{i\alpha} \psi_n^{(\tau)}(i),
  \end{split}
\end{equation}
where the $\psi$'s are the eigenfunctions of symmetric matrices that consist of phenomenological data. At this point the Hubbard-Stratonovich transformation can be used to take these squared operators and convert them to linear operators using the equation
\begin{equation}
  e^{-\frac{1}{2}\lambda_n O_n^2\Delta t} = \left( \frac{\Delta t |\lambda_n|}{2\pi} \right)^{1/2} \int\limits_{-\infty}^{\infty} dx e^{-\frac{1}{2}\Delta t |\lambda_n|x^2 - \Delta t s \lambda_n O_n x}.
\end{equation}
Notice how the $O_n$ term goes from quadratic to linear. This transformation is what gives the method the name Auxiliary Field.
