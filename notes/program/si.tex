\documentclass[12pt]{article}
\usepackage[margin=2.7cm]{geometry}
\usepackage{amsmath}
\usepackage{amssymb}
\begin{document}
\title{Three-body potential notes}
\maketitle
\section{Notation}
\begin{itemize}
\item
Label the orbital states as $|k\rangle$ with a $k$ index.
\item
The position and spin state
of particle $i$ is written $|\vec r_i$ $s_i\rangle$.
\item
We store a particle
spin state as the 4 numbers
$\langle p\uparrow|s_i\rangle$,
$\langle p\downarrow|s_i\rangle$,
$\langle n\uparrow|s_i\rangle$,
$\langle n\downarrow|s_i\rangle$, which is written as
$\langle s|s_i\rangle$ with $s$ running from 1 to 4 respectively.
\item
We use the matrix identity 
${\rm det} S^{-1}S' = \frac{{\rm det} S'}{{\rm det}S}$ to calculate
the determinant of the matrix $S'$ when it has only a small number
of columns diffrent from $A$. The unchanged columns give $1$ on the
diagonal and $0$ on the off diagonals so that the determinant of $C$
with $C=S^{-1}S'$ is given by the determinant of the determinant
of the matrix $C_{mn}$ where $m$ and $n$ only take values of the changed
columns of $S'$.
\end{itemize}

\section{Calculations}
Here we deal with the case where the operators only change the spin state
of the particles. The Slater matrix is
\begin{equation}
S_{ki} = \langle k|\vec r_i s_i\rangle =
\sum_{s=1}^4\langle k|\vec r_i s\rangle \langle s|s_i\rangle
\end{equation}
so a general matrix element of $S'$ will be a linear combination of
the orbital matrix elements
$\langle k|\vec r_i s\rangle$. We therefore precompute
\begin{eqnarray}
{\rm sxmallz(j,s,i)} &=& \sum_k S^{-1}_{jk} \langle k|\vec r_i, s\rangle
\end{eqnarray}
where $s$ runs from 1 to 4. $j$ and $i$ run from 1 to the number of
particles $A$. This quantity is then used for all subsequent determinant
calculations. The original reason for having the spin as the middle index
was to exploit the matrix multiply form.
However, we can get a useful speed up by having the spin
variable leftmost so that it has unit stride in inner loops, so
we transpose this and write
\begin{eqnarray}
{\rm sxz(s,i,j)} &=&
{\rm sxmallz(j,s,i)} = \sum_k S^{-1}_{jk} \langle k|\vec r_i, s\rangle
\end{eqnarray}

The 2- and 3-body potentials are written in terms of products of
linear combinations of the 15 operators for each particle,
$\sigma_{\alpha i}$,
$\tau_{\gamma i}$,
$\sigma_{\alpha i}\tau_{\gamma i}$. We label these $O^p_i$, with $p$ running
from 1 to 15, and greek letters giving the cartesian components.
Since these were the only operators in previous versions of the code,
we calculated
\begin{eqnarray}
{\rm sxmall(j,p,i)} &=& \sum_k S^{-1}_{jk} \langle k|O^p_i|\vec r_i, s_i\rangle
\nonumber\\
&=& \sum_{s=1}^4 \sum_k S^{-1}_{jk} \langle k|\vec r_i s\rangle 
\langle s|O^p_i|s_i\rangle \,.
\end{eqnarray}
In the code this is typically calculated as
\begin{eqnarray}
{\rm spx(s,p,i)} &=& \langle s|O^p_i|s_i\rangle
\nonumber\\
{\rm sxmall(j,p,i)} &=& \sum_s {\rm sxmallz(j,s,i)~ spx(s,p,i)} \,.
\end{eqnarray}

\section{Correlated wave function}
The correlated wave function has a sum of pair wise correlations.
Each pairwise correlation is written as a sum of 12 (or 15, see below)
products of two single particle operators. As in the code, these
will be labeled $i$ and $j$. In order to use the same single determinant
code to calculate the energy, we need the equivalent of
${\rm sxmallz(m,s,n)}$ for the ``Slater matrix''
\begin{equation}
S''_{km} = \left \{
\begin{array}{cc}
S_{km} & m \neq i\ {\rm or}\ j\\
\langle k|{\cal O}^p_m|\vec r_m s_m\rangle & m = i\ {\rm or}\ m = j
\end{array}
\right . \,.
\end{equation}
We require the inverse of this matrix $S''^{-1}$ multiplied by the
all possible ``new orbitals'' which will be
\begin{equation}
\label{eq6}
S''_{km}(s,p)= \left \{
\begin{array}{cc}
\langle k|\vec r_m s\rangle & m\neq i\ {\rm or}\ j\\
\langle k|{\cal O}^p_m|\vec r_m s\rangle &  {\rm otherwise}\\
\end{array}
\right . \,.
\end{equation}
In the code we have
\begin{equation}
\label{eq7}
{\rm sinvijz(s,n,m,p)}=\sum_k S''^{-1}_{mk} S''_{kn}(s,p)
\end{equation}
which is the equivalent of ${\rm sxz(s,n,m)}$ for the $p$ pair operator.
${\rm sinvijz(s,n,m,p)}$ is defined inside a pair loop and is recalculated
for each $ij$ pair. We calculate it and the ratio of determinants
${\rm det} S''/{\rm det}S$. Handing this array to the usual 2- or 3-body
single determinant local energy routine, then hands back the correlated
energy evaluated in the Slater determinant state $S''$, divided by the
determinant of $S''$. Scaling this with the ratio of the determinants
and the pair jastrow function gives the energy contribution for this
part of the correlation operator.

To calculate
$\sum_k S''^{-1}_{mk} S''_{kn}(s,p)$, we can write the usual expression
for the updated inverse when two columns have been changed
in terms of the old inverse. As shown below, multiplying this expression
times $S''_{kn}(s,p)$ shows that all of the terms can be calculated from
${\rm sxmallz(m,s,n)}$ without using the old inverse matrix directly.

Looking at the case for a particular $p$, $i$, and $j$, along with
column $m$ changing. If
$m$ is not equal to $i$ or $j$, i.e.
\begin{equation}
S'''_{kn} = \left \{
\begin{array}{cc}
S_{kn} & n \neq i,j,\ {\rm or}\ m\\
\langle k|{\cal O}^p_m|\vec r_n s_n\rangle & n = i\ {\rm or}\ n = j\\
a_k  & n=m\\
\end{array}
\right . \,,
\end{equation}
where $a_k$ is an arbitrary new column. We can calculate the updated inverse from
\begin{eqnarray}
\frac{{\rm det} S''}{{\rm det} S} &=& {\rm det} S^{-1} S''
\nonumber\\
\frac{{\rm det} S'''}{{\rm det} S} &=& {\rm det} S^{-1} S'''
\nonumber\\
\frac{{\rm det} S'''}{{\rm det} S''} &=& \sum_k S''^{-1}_{mk} a_k
\end{eqnarray}
In this case, 
\begin{equation}
\label{eq10}
\frac{{\rm det} S'''}{{\rm det} S} = {\rm det} S^{-1} S'''
= {\rm det} \left (
\begin{array}{ccc}
\sum_k S^{-1}_{ik}\langle k|{\cal O}_i^p|\vec r_i s_i\rangle
&\sum_k S^{-1}_{ik}\langle k|{\cal O}_j^p|\vec r_j s_j\rangle
&\sum_k S^{-1}_{ik}a_k\\
\sum_k S^{-1}_{jk}\langle k|{\cal O}_i^p|\vec r_i s_i\rangle
&\sum_k S^{-1}_{jk}\langle k|{\cal O}_j^p|\vec r_j s_j\rangle
&\sum_k S^{-1}_{jk}a_k\\
\sum_k S^{-1}_{mk}\langle k|{\cal O}_i^p|\vec r_i s_i\rangle
&\sum_k S^{-1}_{mk}\langle k|{\cal O}_j^p|\vec r_j s_j\rangle
&\sum_k S^{-1}_{mk}a_k\\
\end{array}
\right )
\end{equation}
For the case that $m=j$, we have
\begin{equation}
\frac{{\rm det} S'''}{{\rm det} S} = {\rm det} S^{-1} S'''
= {\rm det} \left (
\begin{array}{cc}
\sum_k S^{-1}_{ik}\langle k|{\cal O}_i^p|\vec r_i s_i\rangle
&\sum_kS^{-1}_{ik}a_k\\
\sum_kS^{-1}_{jk}\langle k|{\cal O}^p_i|\vec r_i s_i\rangle
&\sum_kS^{-1}_{jk}a_k\\
\end{array}
\right ) \,.
\end{equation}

In the code once a particular pair $i$ and $j$ is chosen, we write
\begin{eqnarray}
{\rm sxi(s,m,p)} &=& \sum_k S^{-1}_{mk}\langle k|{\cal O}_i^p|\vec r_i s\rangle
\nonumber\\
{\rm di(m,p) } &=& \sum_k S^{-1}_{mk}\langle k|{\cal O}_i^p|\vec r_i s_i\rangle
\nonumber\\
{\rm sxj(s,m,p)} &=& \sum_k S^{-1}_{mk}\langle k|{\cal O}_j^p|\vec r_j s\rangle
\nonumber\\
{\rm dj(m,p) } &=& \sum_k S^{-1}_{mk}\langle k|{\cal O}_j^p|\vec r_j s_j\rangle
\end{eqnarray}

Eq. \ref{eq10} can be written as
\begin{equation}
\frac{{\rm det} S'''}{{\rm det} S} =
{\rm det} \left (
\begin{array}{ccc}
{\rm di(i,p)} & {\rm dj(i,p)} &\sum_k S^{-1}_{ik}a_k\\
{\rm di(j,p)} & {\rm dj(j,p)} &\sum_k S^{-1}_{jk}a_k\\
{\rm di(m,p)} & {\rm dj(m,p)} &\sum_k S^{-1}_{mk}a_k\\
\end{array}
\right )
\end{equation}

Combining these gives the updated inverse
\begin{equation}
S''^{-1}_{mk}=
\frac{1}{{\rm di(i,p)dj(j,p)-di(j,p)dj(i,p)}}
\left \{
\begin{array}{cc}
\begin{aligned}
&\left [{\rm di(i,p)dj(j,p)-di(j,p)dj(i,p)}\right ] S^{-1}_{mk} \\
-&\left [{\rm di(i,p)dj(m,p)-di(m,p)dj(i,p)}\right ] S^{-1}_{jk} \\
+&\left [{\rm di(j,p)dj(m,p)-di(m,p)dj(j,p)}\right ] S^{-1}_{ik}
\end{aligned}
& m \neq i, m \neq j\\
{\rm dj(j,p)}S^{-1}_{ik} -{\rm dj(i,p)}S^{-1}_{jk} & m=i\\
{\rm di(i,p)}S^{-1}_{jk}-{\rm di(j,p)}S^{-1}_{ik} & m=j\\
\end{array}
\right .
\end{equation}

As noted above, we want Eq. \ref{eq7}, so multiplying the inverse
by $S''_{kn}(s,p)$ given in Eq. \ref{eq6}, and summing over $k$,
the $\sum_k S^{-1}_{mk}S''_{kn}(s,p)$
become sxz terms if $n \neq i$ or $n \neq j$, and sxi or sxj terms
if $n=i$ or $n=j$. At this point, the original inverse is no longer
needed and the result can be built up from the original sxmallz or
sxz array.

\section{Operator breakup}
We rewrite the $v_6$
2-body correlations and potentials as the sum of 15  (or 12
if there are no terms without $\vec \tau_i \cdot\vec \tau_j$ factors)
operator products. That is we write these terms for pair $ij$ as
\begin{equation}
\begin{split}
&\sum_{\alpha\beta} \sigma_{i\alpha} V^\sigma_{\alpha\beta}(ij)\sigma_{j\beta}
+\sum_{\alpha\beta\gamma} \sigma_{i\alpha} V^{\sigma\tau}(ij)
_{\alpha\beta}\sigma_{j\beta}\tau_{i\gamma}\tau_{j\gamma}
+\sum{\gamma}
V^\tau(ij) \tau_{i\gamma}\tau_{j\gamma}
\\
&= \sum_{p=1}^{3} V^{\sigma p}(ij) {\cal O}^\sigma_p(i){\cal O}^\sigma_p(j)
+\sum_{p=1}^{9} V^{\sigma\tau p}(ij)
{\cal O}^{\sigma\tau}_p(i){\cal O}^{\sigma\tau}_p(j)
+\sum_{p=1}^{3} V^{\tau p}(ij) {\cal O}^\tau(i){\cal O}^\tau(j)
\end{split}
\end{equation}
{\em fix this. The notation is bad and wrong.}

\section{Application to the pairwise sum correlation}


\end{document}

