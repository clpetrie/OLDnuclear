\documentclass{beamer}
\usepackage{amsmath}
%\usepackage{beamerthemesplit} % new 
\usetheme{Madrid}
\usefonttheme[onlymath]{serif}
\setbeamertemplate{frametitle}[default][center] %center slide titles

\begin{document}
\title{Correlated Trial Wave Function}
\author{Cody Petrie} 
\date{\today} 

%commands
\newcommand{\Oi}{\mathcal{O}_{i}}
\newcommand{\Oij}{\mathcal{O}_{ij}}
\newcommand{\Okl}{\mathcal{O}_{kl}}
\newcommand{\Oijp}{\mathcal{O}^p_{ij}}
\newcommand{\Oklp}{\mathcal{O}^p_{kl}}
\newcommand{\ket}[1]{\left| #1 \right>}
\newcommand{\bra}[1]{\left< #1 \right|}
\newcommand{\braket}[2]{\left< #1 | #2 \right>}
\newcommand{\ketbra}[2]{\left| #1 \right> \left< #2 \right|}
\newcommand{\taui}{\bm{\tau}_i}
\newcommand{\tauj}{\bm{\tau}_j}
\newcommand{\sigmai}{\bm{\sigma}_i}
\newcommand{\sigmaj}{\bm{\sigma}_j}
\newcommand{\rij}{\hat{r}_{ij}}
\newcommand{\sigmaia}{\sigma_{i\alpha}}
\newcommand{\sigmaib}{\sigma_{i\beta}}
\newcommand{\tauig}{\tau_{i\gamma}}
\newcommand{\sigmaja}{\sigma_{j\alpha}}
\newcommand{\sigmajb}{\sigma_{j\beta}}
\newcommand{\taujg}{\tau_{j\gamma}}
\newcommand{\tauij}{\taui \cdot \tauj}
\newcommand{\sigmaij}{\sigmai \cdot \sigmaj}
\newcommand{\mycolor}[1]{\textit{\textcolor{red}{#1}}}
\newcommand{\longsi}{s_1, \ldots, s_{i-1} , s, s_{i+1}, \ldots, s_A}
\newcommand{\longsij}{s_1, \ldots, s_{i-1} , s, s_{i+1}, \ldots, s_{j-1}, s', s_{j+1}, \ldots ,s_A}
\newcommand{\Ot}{\mathcal{O}^\tau_{n\alpha}}
\newcommand{\Os}{\mathcal{O}^\sigma_{n}}
\newcommand{\Ost}{\mathcal{O}^{\sigma\tau}_{n\alpha}}
\newcommand{\detr}{\mathrm{det}}
\newcommand{\RS}{\mathrm{RS}}

%start slides
\frame{\titlepage} 

\frame{\frametitle{Correlated Trial Wave Function}
\uncover<1,2,3>{
   \begin{equation}
      \braket{\Psi_T}{\RS} = \bra{\Phi} \prod\limits_{i<j}\left[f_c(r_{ij})\left[1+\sum\limits_{i<j,p}f_p(r_{ij})\Oijp\right]\right] \ket{\RS}
   \end{equation}
}
\uncover<2,3>{
   \begin{equation}
      \braket{\Psi_T}{\RS} = \bra{\Phi} \left[\prod\limits_{i<j}f_c(r_{ij})\right]\left[1+\sum\limits_{i<j,p}f_p(r_{ij})\Oijp\right] \ket{\RS}
   \end{equation}
}
\uncover<3>{
   \begin{equation}
   \begin{split}
      \braket{\Psi_T}{\RS} = \bra{\Phi} \left[\prod\limits_{i<j}f_c(r_{ij})\right]\left[1+\sum\limits_{i<j,p}f_p(r_{ij})\Oijp \right. \\
         \left.+\sum\limits_{i<j,p}\sum\limits_{\substack{k<l \\ \mathrm{ind pair?}}}f_p(r_{ij})\Oijp f_p(r_{kl})\Oklp\right] \ket{\RS}
   \end{split}
   \end{equation}
}
}

\frame{\frametitle{Independent Pair}
\begin{itemize}
   \item Independent pair sum looks like this.
   \begin{equation}
      \sum\limits_{\substack{k<l \\ \mathrm{ind pair}}} \rightarrow \sum\limits_{\substack{k<l \\ k,l \ne i,j}}
   \end{equation}
   \item This will give us insight into how much the correlations from different sets of particle effect the energy.
\end{itemize}
}

\frame{\frametitle{Method}
\begin{itemize}
   \item Correlation terms are currently calculated in the code as
   \begin{equation}
      \frac{\braket{\Psi_T}{\RS}}{\braket{\Phi}{RS}} = \mathrm{sum(d2b*f2b)}
   \end{equation}
   where the d2b and f2b look like
   \begin{equation}
      \mathrm{d2b}(s,s',ij) = \frac{\braket{\Phi}{R,\longsij}}{\braket{\Phi}{R,S}}
   \end{equation}
   \begin{equation}
      \mathrm{f2b}(s,s',ij) = \bra{s,s'} \Oijp \ket{s_i,s_j}
   \end{equation}
\end{itemize}
}

\frame{\frametitle{Method}

}

\end{document}

