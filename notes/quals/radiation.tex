\documentclass[12pt]{article}
\usepackage[margin=1in]{geometry}
\usepackage{amssymb}
\usepackage{amsmath}
\usepackage{color}
\usepackage{graphicx}

%my commands
\newcommand{\wavr}{\left|\mathbf{r}-\mathbf{r}'\right|}
\newcommand{\br}{\mathbf{r}}
\newcommand{\tret}{t-\frac{\wavr}{c}}

\title{Notes on Radiation}
\author{Cody L. Petrie}

\begin{document}
\maketitle

\section*{Radiation from arbitrary source}
Let's start with the retarted sources which give us the scalar and vector potentials
\begin{align}
   \mathbf{\Phi}(\br,t) &= \frac{1}{4\pi\epsilon_0} \int d^3r' \frac{\rho(\br,\tret)}{\wavr} \\
   \mathbf{A}(\br,t) &= \frac{\mu_0}{4\pi} \int d^3r' \frac{\mathbf{J}(\br,\tret)}{\wavr}.
\end{align}
The first approximation for radiation is that the source is localized (also meaning that the radiation zone is far from the source $r \gg r'$. When calculating

\end{document}
