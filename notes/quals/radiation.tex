\documentclass[12pt]{article}
\usepackage[margin=1in]{geometry}
\usepackage{amssymb}
\usepackage{amsmath}
\usepackage{color}
\usepackage{graphicx}

%my commands
\newcommand{\wavr}{\left|\mathbf{r}-\mathbf{r}'\right|}
\newcommand{\br}{\mathbf{r}}
\newcommand{\tret}{t-\frac{\wavr}{c}}
\newcommand{\pd}{\dot{\mathbf{p}}}
\newcommand{\pdd}{\ddot{\mathbf{p}}}

\title{Notes on Radiation}
\author{Cody L. Petrie}

\begin{document}
\maketitle

\section*{Radiation from arbitrary source}
Let's start with the retarted sources which give us the scalar and vector potentials
\begin{align}
   \mathbf{\Phi}(\br,t) &= \frac{1}{4\pi\epsilon_0} \int d^3r' \frac{\rho(\br',\tret)}{\wavr} \label{equ:genphi} \\
   \mathbf{A}(\br,t) &= \frac{\mu_0}{4\pi} \int d^3r' \frac{\mathbf{J}(\br',\tret)}{\wavr}. \label{equ:genj}
\end{align}
The first approximation for radiation is that the source is localized (also meaning that the radiation zone is far from the source $r \gg r'$). Under this assumption we get,
\begin{align}
   \wavr &\approx r - \frac{\br \cdot \br'}{r} \label{equ:approxwr} \\
   \frac{1}{\wavr} &\approx \frac{1}{r} + \frac{\br \cdot \br'}{r^3}, \label{equ:approxwrm1}
\end{align}
using the Fourier Transform
\begin{equation}
   f(\br+\mathbf{a}) \approx f(\br) + \mathbf{a} \cdot \nabla f(\br) = f(\br) + a_i \partial_i f(\br).
\end{equation}
Now lets use the fact that the sources have harmonic time dependence.
\begin{equation}
   \mathbf{J}(\br',t) = \mathbf{J}(\br')e^{-i\omega t}
\end{equation}
With this we can now approximate the vector potential by
\begin{equation}
   \mathbf{J}(\br',\tret) \approx \mathbf{J}(\br')e^{-i\omega t + i\omega r/c - i\omega \frac{\hat{r} \cdot \br'}{c}}.
\end{equation}
Now we can plug this into equation \ref{equ:genj}, and dropping terms with higher order terms that $1/r$ to get,
\begin{equation}
   \mathbf{A}(\br,t) = \frac{\mu_0e^{-i\omega(t-r/c)}}{4\pi r} \int d^3r' \mathbf{J}(\br')e^{-i\omega \frac{\hat{r} \cdot \br}{c}}.
   \label{equ:localizeda}
\end{equation}
Now is we assume that the region of sources is small compared to the wavelength, $c/\omega \gg r'$,  then we can make the \textbf{dipole approximation}, $e^{-i\omega \hat{r}\cdot\br'/c} \approx 1$, which when pluggen into equation~\ref{equ:localizeda} gives
\begin{equation}
   \mathbf{A}(\br,t) = \frac{\mu_0e^{-i\omega(t-r/c)}}{4\pi r} \int d^3r' \mathbf{J}(\br').
   \label{equ:dipapproxa}
\end{equation}
Now let's go on an aside to explore the $\mathbf{J}(\br')$. \\
\textbf{Claim:} $\int d^3r' \mathbf{J}(\br') = \frac{d\mathbf{p}}{dt}$ where
\begin{equation}
   \mathbf{p}(\br',t) = \int d^3r' \br'\rho(\br',t)
\end{equation}
is the electric dipole moment. Now let's find the integral of $\mathbf{J}(\br')$ in terms of this dipole moment. Let's start with the time derivative of the dipole.
\begin{align}
   \frac{d\mathbf{p}}{dt} &= \frac{d}{dt}\int d^3r' \br'\rho(\br',t) \\
   &= \int d^3r' \br' \frac{\partial}{\partial t}\rho(\br',t) \\
   &= -\int d^3r' \br' \nabla \cdot \mathbf{J}(\br',t),
\end{align}
where we have used the continuity equation
\begin{equation}
   \nabla \cdot \mathbf{J(\br',t)} = -\frac{\partial}{\partial t} \rho(\br',t).
\end{equation}
Now expand the $\br' \nabla \cdot \mathbf{J}(\br',t)$ term.
\begin{align}
   \nabla \cdot r_i\mathbf{J} &= (\nabla r_i) \cdot \mathbf{J} + r_i\nabla\cdot\mathbf{J} \\
   &= \hat{r_i} \cdot \mathbf{J} + r_i\nabla\cdot\mathbf{J}
\end{align}
If you sum over all possible $r_i$'s you get
\begin{equation}
   \br'\nabla\cdot\mathbf{J} = \nabla\cdot (r_x\mathbf{J}+r_y\mathbf{J}+r_z\mathbf{J}) - \mathbf{J}
\end{equation}
However when the second term is integrated over the volume it can be turned into a surface integral with the divergence theorem, but $\mathbf{J}$ is zero on the surface to we get that
\begin{equation}
   \frac{d\mathbf{p}}{dt} = \int d^3r' \mathbf{J}.
\end{equation}
Now applying this to equation~\ref{equ:dipapproxa} we get
\begin{align}
   \mathbf{A}(\br,t) &= \frac{\mu_0}{4\pi r} \frac{d}{dt}\mathbf{p}e^{-i\omega(t-r/c)} \\
   &= \frac{\mu_0}{4\pi r} \pd(t-r/c),
\end{align}
which is the standard result. I'm not going to worry about finding $\Phi(\br,t)$ because if
\begin{align}
   \mathbf{E} &= \mathbf{E}_0e^{-i(\mathbf{k}\cdot\br - \omega t)} \\
   \mathbf{B} &= \mathbf{B}_0e^{-i(\mathbf{k}\cdot\br - \omega t)},
\end{align}
it can be shown with Maxwell's equations that $\mathbf{E}_0 = -c \hat{k}\times\mathbf{B}_0$, so we only need to solve for $\mathbf{B}$ using $\mathbf{B} = \nabla\times\mathbf{A}$.
\begin{align}
   \mathbf{B} &= \nabla\times\mathbf{A} \\
   &= \frac{\mu_0}{4\pi} \nabla\times\frac{\pd(t-r/c)}{r} \\
   &= \frac{\mu_0}{4\pi r} \nabla\times\pd(t-r/c)
\end{align}
Where the chain rule was applied here but the derivative of $1/r$ gave a term that went as $1/r^2$ so we ignored it. Now this cross product can be done if we remember the product rule $\nabla\times(f\mathbf{A}) = f(\nabla\times\mathbf{A}) - \mathbf{A}\times(\nabla f)$. Thus $\mathbf{B}$ becomes
\begin{align}
   \mathbf{B}(\br,t) &= -\frac{\mu_0}{4\pi r}\left[e^{-i\omega(t-r/c)}(\nabla\times\pd) - \pd\times(\nabla e^{-i\omega(t-r/c)}) \right] \\
   &= \frac{\mu_0}{4\pi r} \pd\times(\nabla e^{-i\omega(t-r/c)}) \\
   &= \frac{\mu_0}{4\pi r} \frac{i\omega}{c}\hat{r}\times\pd e^{-i\omega(t-r/c)}.
\end{align}
but since we said that $\pd(t-r/c) = \pd e^{-i\omega(t-r/c)}$ we can say that
\begin{equation}
   \frac{d}{dt_{ret}}\pd = \pdd = -i\omega\pd,
\end{equation}
which we can use to give us
\begin{equation}
   \mathbf{B}(\br,t) = -\frac{\mu_0}{4\pi rc}\hat{r}\times\pdd(t-rc).
\end{equation}
Now we can find $\mathbf{E}$ to be
\begin{equation}
   \mathbf{E}(\br,t) = \frac{\mu_0}{4\pi r} \hat{r}\times\hat{r}\times\pdd(\br,t-r/c),
\end{equation}
using the fact that $\mathbf{E}_0 = -c \hat{k}\times\mathbf{B}_0$, and noting that $\hat{k}=\hat{r}$.

\section*{Power Radiated}
The Poynting vector gives us the the energy that is radiated per time per unit area.
\begin{equation}
   \mathbf{S}=\frac{1}{\mu_0} \mathbf{E} \times \mathbf{B}
\end{equation}
This points in the $\hat{k}$ of $\hat{r}$ direction but if you use the relation used above you can get the magnitude in terms of only $\mathbf{E}$ or $\mathbf{B}$.
\begin{align}
   \mathbf{S} &= \frac{c}{\mu_0} |\mathbf{B}|^2\hat{r} \\
   &= \frac{\mu_0}{16\pi^2 c r^2} |\hat{r}\times\pdd|^2 \hat{r}
\end{align}
We can now get the total radiated power by integrating this over an entire sphere since the Poynting vector has units of power per area. Let's let $\pdd = \ddot{p}\hat{z}$ so that $|\hat{r}\times\pdd|^2 = |\pdd|^2\sin^2 \theta$, so we get
\begin{equation}
   \mathbf{S} = \frac{\mu_0}{16\pi^2 c r^2} |\pdd|^2\sin^2 \theta \hat{r},
\end{equation}
where $\theta$ is the typical spherical angle. Now let's integrate this over the full sphere to get the total power radiated.
\begin{align}
   P &= \int r^2 \mathbf{\Omega}\cdot\mathbf{S} \\
   &= \int r^2 \Omega S \hat{r}\cdot\hat{r} \\
   &= \int r^2\sin\theta d\theta d\phi \frac{\mu_0}{16\pi^2cr^2}|\pdd|^2\sin^2\theta \\
   &= \int_0^{2\pi} \int_0^\pi d\phi d\theta \frac{\mu_0}{16\pi^2c} |\pdd|^2 \sin^3\theta \\
   &= \int_0^{2\pi} d\theta \frac{\mu_0}{8\pi c} |\pdd|^2 \sin^3\theta
\end{align}
Now we just need to do the $\theta$ integral. To do this use a u-sum with $u=\cos\theta$ so that $du=-\sin\theta d\theta$.
\begin{align}
   \int_0^\pi d\theta \sin^3\theta &= \int_0^\pi d\theta \sin\theta(1-\cos^2\theta) \\
   &= -\int_1^{-1} du (1-u^2) \\
   &= \left. 1-\frac{1}{3}u^3\right|_{-1}^1 \\
   &= \frac{4}{3}
\end{align}
This then gives us for the total power radiated
\begin{equation}
   P = \frac{\mu_0}{6\pi c} |\pdd|^2
\end{equation}

\section*{Time Averaged Power Radiated/Larmor Formula}
Now to get the time averaged power let's assume the form $\mathbf{p}(t)) = p\hat{\mathbf{z}}e^{i\omega t}$ for the dipole moment so that we get $\pdd = -\omega^2p\hat{\mathbf{z}}e^{i\omega t}$, where the period is $T=2\pi/\omega$. To time agerage we integrate over a period and divide by the period.
\begin{align}
   \bar{P} &= \frac{1}{T} \int_0^T dt P \\
   &= \frac{\omega}{2\pi} \int_0^{2\pi/\omega} dt \frac{\mu_0\omega^4p^2}{6\pi c} \\
   &= \frac{\omega}{2\pi} \frac{\mu_0\omega^4p^2}{12\pi c} t \bigg|_0^{2\pi/\omega} \\
   &= \frac{\mu_0\omega^4p^2}{12\pi c}
\end{align}
Now if we say that $p=Qd$ we can write this as
\begin{equation}
   \bar{P} = \frac{\mu_0\omega^4p^2}{12\pi c} = \frac{\mu_0\omega^4Q^2d^2}{12\pi c}.
\end{equation}
This is the Larmor formula. Notice the strong dependence on the frequency $\omega$.

\end{document}
