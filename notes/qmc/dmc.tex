\section*{Diffusion Monte Carlo}
This explanation also draws heavily from \cite{foulkes2001}, but also draws a good deal from \cite{kosztin1996}. The time dependent Schr\"odinger equation is
\begin{equation}
  \hat H \Psi = i \hbar \frac{\partial \Psi}{\partial t}
\end{equation}
where
\begin{equation}
  \hat H = -\frac{\hbar^2}{2m}\frac{\partial^2\Psi}{\partial t^2} + V.
\end{equation}
To get the imaginary time Schr\"odinger equation make the substitution $\tau=i t/\hbar$, or $t=-i \tau / \hbar$ to get
\begin{equation}
  \hat H \Psi = - \frac{\partial \Psi}{\partial \tau}.
\end{equation}
Notice that this is the diffusion equation (thus ``diffusion'' Monte Carlo), and thus the solutions to this consists of exponentials
\begin{equation}
  \Psi(r,\tau) = \sum\limits_{n=0}^\infty c_n \phi_n(r) e^{- \tau E_n}.
\end{equation}
We now perform an energy shift to setting $V=V-E_0$ and $E_n=E_n-E_0$ to get
\begin{equation}
  \Psi(r,\tau) = \sum\limits_{n=0}^\infty c_n \phi_n(r) e^{- \tau (E_n-E_0)}
\end{equation}
Now here is a key part, as you let $\tau \rightarrow \infty$ you get that $\Psi(r,\tau) \rightarrow \phi_0(r,\tau)$, which is the ground state wave function. This is because $E_n-E_0$ for $n>0$ is a large number and essentially goes to zero.
\begin{equation}
  \begin{split}
    \lim_{\tau \to \infty} \Psi(r,\tau) &= \lim_{\tau \to \infty} \sum\limits_{n=0}^\infty c_n \phi_n(r) e^{- \tau (E_n-E_0)} \\
    &= c_0 \phi_0(r) + \lim_{\tau \to \infty} \sum\limits_{n=1}^\infty c_n \phi_n(r) e^{- \tau (E_n-E_0)} \\
    &= c_0 \phi_0(r)
  \end{split}
\end{equation}

To determine how to diffuse each walker and how to do the birth/death algorithm, lets look at the propagator (I'm just going to do the 1D case here) $\left| \psi(\tau) \right> = e^{-(H-E_r)\tau} \left| \psi(0) \right>$. The propagator, often called the Green's function, and is given by
\begin{equation}
  \begin{split}
    \mathbf{G}(\R) &= \left<\R\right| e^{-(H-E_r)\tau} \left|\R'\right>\\
    &= \left<\R\right| e^{-(V-Er)\Delta\tau/2} e^{(\frac{p^2}{2m})\Delta\tau} e^{-(V-Er)\Delta\tau/2} \left|\R'\right>\\
    &= e^{-\Delta\tau\left(V(\R)+V(\R')-2E_r\right)/2} \left<\R\right| e^{(\frac{p^2}{2m})\Delta\tau} \left|\R'\right>.
  \end{split}
\end{equation}
The part including the potential gives
\begin{equation}
  P = \exp{(-\Delta\tau\left(V(\R)+V(\R')-2E_r\right)/2)}.
  \label{equ:weight}
\end{equation}
We need to take the Fourier trannsform of the kinetic part. This gives
\begin{align}
  \bra{\R} \exp{(-\frac{p^2 \tau}{2m})} \ket{\R'} &= \int_{-\infty}^\infty dp \braket{\R}{\mathbf{p}} \exp{(-\frac{p^2 \tau}{2m})} \braket{\mathbf{p}}{\R'} \\
  &= \int_{-\infty}^\infty dp ~ e^{i\mathbf{p} \cdot \R} e^{-\frac{p^2 \tau}{2m}} e^{-i \mathbf{p} \cdot \R'} \\
  &= \sqrt{\frac{2m\pi}{\tau}} \exp{\left( -\frac{m(\R - \R')^2}{2\tau} \right)}
  \label{equ:sample}
\end{align}
Equation \ref{equ:weight} is used to do the birth/death algorithm since it acts as a renormalization term and we sample from equation \ref{equ:sample} to move the walkers at each step since it is the kinetic term.

\subsection*{Procedure}
The steps for performing DMC are shown below.
\begin{enumerate}
\item \textbf{Initialize:}
  \begin{enumerate}
  \item Randomly choose a distribution of walkers from some trial wave function $\Psi_T$.
  \item The reference energy is set to something like the energy that you get from VMC, or the energy of the trial wave function.
  \end{enumerate}

\item \textbf{Imaginary time propagation}
  \begin{enumerate}
  \item Advance the time by $\Delta \tau$.
  \item Move each walker according to Gaussian distribution with mean around the current walker and with standard deviation equal to
    \begin{equation}
      \sigma = \sqrt{\frac{\hbar \Delta \tau}{m}}
    \end{equation}
  \item The reference energy is then reset to the value
    \begin{equation}
      E_{R,i} = E_{R,i-1} + \frac{\hbar}{\Delta \tau} \left( 1-\frac{N_i}{N_{i-1}}\right)
    \end{equation}
  \item The birth/death process is carried out. The integer part of the parameter $m$ is gives the number of walkers that process from 1 walker, where
    \begin{equation}
      m = \mathrm{int} (P+\eta),
    \end{equation}
    where $P$ is given by equation~\ref{equ:weight} and $\eta$ is drawn from a uniform distribution on $[0,1]$. A maximum of two walkers can come from any walker \cite{kosztin1996}.
  \end{enumerate}

\item The ground state is then given by the average of successive iterations of $E_R$. The ground state wave function is then given by the distribution of walkers (histogram).
\end{enumerate}

\subsection*{Importance Sampling}
Importance sampling is used in \cite{foulkes2001} to improve the stability of DMC. This is done by multiplying the time dependent imaginary time schr\"odinger equation,
\begin{equation}
  -\partial_t \Phi(\R,t) = (\hat{H}-E_T) \Phi(\R,t),
\end{equation}
by $\Psi_T$ on both sides and define $f(\R,t) \equiv \Phi(\R,t) \Psi_T(\R)$. This gives us
\begin{equation}
  \begin{split}
    -\partial_t f(\R,t) &= \Psi_T(\R) \hat{H} \Phi(\R,t) -E_T f(\R,t) \\
    &= -\Psi_T \frac{1}{2} \nabla^2 \Phi + (V-E_T)f \\
    &= -\frac{1}{2}\nabla^2f + \frac{1}{2}\Phi\nabla^2\Psi_T + \frac{1}{2}2(\nabla\Phi)(\nabla\Psi_T) + \frac{1}{2}\Phi\nabla^2\Psi_T - \frac{1}{2}\Phi\nabla\Psi_T - E_Tf + Vf \\
    &= -\frac{1}{2}\nabla^2f + \nabla \cdot (\Phi\nabla\Psi_T) - \frac{1}{2}\Phi\nabla^2\Psi_T-E_Tf+Vf \\
    -\partial_t f(\R,t) &= -\frac{1}{2}\nabla^2f(\R,t) + \nabla \cdot (\mathbf{v_D}f(\R,t)) + (E_L-E_T)f(\R,t),
  \end{split}
\end{equation}

where I have used

\begin{equation}
  \begin{split}
    \nabla^2f &= \Phi\nabla^2\Psi_T + \Psi_T\nabla^2\Phi + 2(\nabla\Phi)(\nabla\Psi_T) \\
    \nabla\cdot(\Phi\nabla\Psi_T) &= (\nabla\Phi)(\nabla\Psi_T) + \Phi\nabla^2\Psi_T \\
    \mathbf{V_D} &= \Psi_T^{-1}\nabla\Psi_T = \nabla \ln(\Psi_T) \\
    E_L &= \Psi_T^{-1}\hat{H}\Psi_T = -\Psi_T^{-1}\frac{1}{2}\nabla^2\Psi_T + V.
  \end{split}
\end{equation}

Now the steps to do this method is described by \cite{foulkes2001} to be
\begin{enumerate}
  \item{Generate walkers from $\left| \Psi_T \right|^2$ distribution.}
  \item{Calculate the local energies and $v_D$ for each walker.}
  \item{Move each walker according to $\R = \R' + \chi + \Delta\tau \mathbf{v_D}(\R')$, where $\chi$ is drawn from a Gaussian with variance $\Delta\tau$ and zero mean.}
  \item{If the trial wave function changes sign move it back to the original position.}
  \item{Accept the step with probability
    \begin{equation}
      P_{\mathrm{accept}}(\R \leftarrow \R') = \mathrm{min}\left[ 1, \frac{G_d(\R' \leftarrow \R, \Delta\tau) \Psi_T(\R)^2}{G_d(\R \leftarrow \R', \Delta\tau) \Psi_T(\R')^2} \right]
    \end{equation}
    where $G_d(\R \leftarrow \R', \Delta\tau)$ is given by
    \begin{equation}
      G_d(\R \leftarrow \R', \Delta\tau) = (2\pi\Delta\tau)^{-3N/2} \exp{\left[ -\frac{[\R-\R'-\Delta\tau\mathbf{v_D}(\R')]^2}{2\Delta\tau} \right]}
    \end{equation}
  }
  \item{Calculate how many walkers will come from each point using
    \begin{equation}
      M_{new} = \mathrm{int}(\eta+\exp\left\{-\Delta\tau[E_L(\R)+E_L(\R')-2E_T/2]\right\}),
    \end{equation}
    where $\eta$ is drawn from a uniform distribution $[0,1]$.
  }
  \item{Accumulate the appropriate quantities to obtain the ground state energy and wave function.}
\end{enumerate}
The steps are then repeated until the ground state energy and wave function are projected out.

The importance sampling does a few things. First the drift velocity pulls walkers in the direction of large $\Psi_T$, and secondly the renormalizing term contains $E_L$ instead of $V$. This decreases the polulation fluctuations.
