\documentclass[12pt]{extarticle}
\usepackage[margin=1in]{geometry}
\usepackage{amssymb}
\usepackage{amsmath}
\usepackage{url}
\usepackage{bm}
\usepackage{color}
\usepackage{fancyvrb}

%My commands
\newcommand{\Oi}{\mathcal{O}_{i}}
\newcommand{\Oij}{\mathcal{O}_{ij}}
\newcommand{\Okl}{\mathcal{O}_{kl}}
\newcommand{\Oijp}{\mathcal{O}^p_{ij}}
\newcommand{\Oklp}{\mathcal{O}^p_{kl}}
\newcommand{\ket}[1]{\left| #1 \right>}
\newcommand{\bra}[1]{\left< #1 \right|}
\newcommand{\braket}[2]{\left< #1 | #2 \right>}
\newcommand{\ketbra}[2]{\left| #1 \right> \left< #2 \right|}
\newcommand{\taui}{\bm{\tau}_i}
\newcommand{\tauj}{\bm{\tau}_j}
\newcommand{\sigmai}{\bm{\sigma}_i}
\newcommand{\sigmaj}{\bm{\sigma}_j}
\newcommand{\tauij}{\taui \cdot \tauj}
\newcommand{\sigmaij}{\sigmai \cdot \sigmaj}
\newcommand{\mycolor}[1]{\textit{\textcolor{red}{#1}}}
\newcommand{\longsi}{s_1, \ldots, s_{i-1} , s, s_{i+1}, \ldots, s_A}
\newcommand{\longsij}{s_1, \ldots, s_{i-1} , s, s_{i+1}, \ldots, s_{j-1}, s', s_{j+1}, \ldots ,s_A}
\newcommand{\bverb}{\begin{Verbatim}[frame=single]}
\newcommand{\everb}{\end{Verbatim}}

\title{Dissection of subroutine corpsi}
\author{Cody L. Petrie}

\begin{document}
\maketitle

The subroutine \texttt{corpsi} I believe stands for ``correlate psi" where psi is the trial wavefunction. I will post the code and try to explain what each section above the comments means.

\bverb
   subroutine corpsi(sp,d1b,d2b,d3b)
\end{Verbatim}
Here the \texttt{d1b, d2b,} and \texttt{d3b} are used to get expectation values like as so
\begin{equation}
  \left< \mathcal{O} \right> = \sum_i\sum_s \alpha_{is}\texttt{d1b(s,i)}.
\end{equation}
The other two are calculated in a similar manner. See my write up on the trial wave function to see details.

\bverb
   complex(kind=r8), intent(in) :: sp(:,:)
   complex(kind=r8), intent(inout):: d1b(:,:),d2b(:,:,:),d3b(:,:,:,:)
   complex(kind=r8) :: fij,f1,fijk
   complex(kind=r8) :: detrat,sxzi(4,npart,npart,15)
   complex(kind=r8) :: sxzj(4,npart,npart),d1,d2,d15(15)
   complex(kind=r8) :: sx15(4,15,npart,npart),sx15j(4,15,npart)
   complex(kind=r8) :: sx15j1(4,15,npart),sx15j2(4,15,npart)
   complex(kind=r8) :: sx15k1(4,15,npart),sx15k2(4,15,npart)
   complex(kind=r8) :: sxzi1(4,npart,npart),di1
   complex(kind=r8) :: sxzj1(4,npart,npart),sxzj2(4,npart,npart)
   complex(kind=r8) :: sxzk(4,npart,npart),dj1,dj2,dk
   integer(kind=i4) :: i,j,ij,iop,is,it,js,k,ks,ijk
   d1b=czero
   d2b=czero
   d3b=czero
   call g1bval(d1b,sxz0,cone+fctau)
   call g2bval(d2b,sxz0,cone+fctau)
   call g3bval(d3b,sxz0,cone+fctau)
\end{Verbatim}
Most of this is just setup but \texttt{g1bval, g2bval,} and \texttt{g3bval} are subroutines used to calculate \texttt{d1b} etc. This part includes the detterminant of the unchanged Slater matrix.

\bverb
   do i=1,npart
      sx15(:,:,:,i)=conjg(opmult(conjg(sxz0(:,i,:))))
   enddo
\end{Verbatim}
The subroutine \texttt{opmult} operates on the spin components of a spinor with all 15 spin operators and returns the new spinor.
\begin{equation}
  \ket{s_i} \rightarrow \Oi\ket{s_i}
\end{equation}
Here the $\Oi$'s in order are 1-3 sx, sy, sz, 4-6 tx, ty, tz, 7-9 sx*(tx, ty, tz), 10-12 sy*(tx, ty, tz), 13-15 sz*(tx, ty, tz).

\bverb
   do i=1,npart
      if (abs(ftau1(i)).le.0.0_r8) cycle
      call sxzupdate(sxzj,d1,sxz0,i,sx15(:,6,:,i),sp(:,i))
      f1=d1*0.25_r8*ftau1(i)
      call g1bval(d1b,sxzj,f1)
      call g2bval(d2b,sxzj,f1)
      call g3bval(d3b,sxzj,f1)
   enddo
\end{Verbatim}
Here somekind of correlation is being added, it looks like a correlation related to the tau operators. \texttt{sxzupdate} is updating the inverse (\texttt{sxz}) and \texttt{g1bval} and the other two are adding on their respective determinants.

\bverb
   ij=0
   do i=1,npart-1
      do iop=1,15
         call sxzupdate(sxzi(:,:,:,iop),d15(iop),sxz0,i,sx15(:,iop,:,i),sp(:,i))
      enddo
      do j=i+1,npart
         ij=ij+1
         if (doft(ij)) then
            do it=1,2
               fij=ft(ij)
               sx15j(:,:,:)=conjg(opmult(conjg(sxzi(:,j,:,3+it))))
               call sxzupdate(sxzj,d2,sxzi(:,:,:,3+it),j,sx15j(:,3+it,:) &
                  ,sp(:,j))
               detrat=d15(3+it)*d2
               fij=detrat*fij
               call g1bval(d1b,sxzj,fij)
               call g2bval(d2b,sxzj,fij)
               call g3bval(d3b,sxzj,fij)
            enddo
         endif
         if (doft(ij).or.doftpp(ij).or.doftnn(ij)) then
            it=3
            fij=ft(ij)
            if (doftpp(ij)) fij=fij+0.25_r8*ftpp(ij)
            if (doftnn(ij)) fij=fij+0.25_r8*ftnn(ij)
            sx15j(:,:,:)=conjg(opmult(conjg(sxzi(:,j,:,3+it))))
            call sxzupdate(sxzj,d2,sxzi(:,:,:,3+it),j,sx15j(:,3+it,:),sp(:,j))
            detrat=d15(3+it)*d2
            fij=detrat*fij
            call g1bval(d1b,sxzj,fij)
            call g2bval(d2b,sxzj,fij)
            call g3bval(d3b,sxzj,fij)
         endif
         if (dofs(ij)) then
            do is=1,3
               sx15j(:,:,:)=conjg(opmult(conjg(sxzi(:,j,:,is))))
               do js=1,3
                  call sxzupdate(sxzj,d2,sxzi(:,:,:,is),j,sx15j(:,js,:),sp(:,j))
                  detrat=d15(is)*d2
                  fij=detrat*fs(is,js,ij)
                  call g1bval(d1b,sxzj,fij)
                  call g2bval(d2b,sxzj,fij)
                  call g3bval(d3b,sxzj,fij)
               enddo
            enddo
         endif
         if (dofst(ij)) then
            do it=1,3
               do is=1,3
                  sx15j(:,:,:)=conjg(opmult(conjg(sxzi(:,j,:,3*is+it+3))))
                  do js=1,3
                     call sxzupdate(sxzj,d2,sxzi(:,:,:,3*is+it+3),j &
                        ,sx15j(:,3*js+it+3,:),sp(:,j))
                     detrat=d15(3*is+it+3)*d2
                     fij=detrat*fst(is,js,ij)
                     call g1bval(d1b,sxzj,fij)
                     call g2bval(d2b,sxzj,fij)
                     call g3bval(d3b,sxzj,fij)
                  enddo
               enddo
            enddo
         endif
      enddo
   enddo
   if (.not.dof3) return !skip 3-body correlation
   do i=1,npart-2
      do is=1,3
         do it=1,3
            call sxzupdate(sxzi1,di1,sxz0,i,sx15(:,3*is+it+3,:,i),sp(:,i))
            do j=i+1,npart-1
               sx15j(:,:,:)=conjg(opmult(conjg(sxzi1(:,j,:))))
               do js=1,3
                  call sxzupdate(sxzj1,dj1,sxzi1,j &
                     ,sx15j(:,3*js+3+levi(1,it),:),sp(:,j))
                  call sxzupdate(sxzj2,dj2,sxzi1,j &
                     ,sx15j(:,3*js+3+levi(2,it),:),sp(:,j))
                  do k=j+1,npart
!maple ijk := simplify(sum((n-l)*(n-l-1)/2,l=1..i-1)+sum(n-l,l=i+1..j-1)+k-j);
                     ijk=(i*(i-1)*(i-3*npart+4))/6 &
                        +((npart-2)*(npart-1)*(i-1))/2-2 &
                        +((2*npart-4-j+1)*(j-2))/2+k
                     sx15k1(:,:,:)=conjg(opmult(conjg(sxzj1(:,k,:))))
                     sx15k2(:,:,:)=conjg(opmult(conjg(sxzj2(:,k,:))))
                     do ks=1,3
                        call sxzupdate(sxzk,dk,sxzj1,k &
                           ,sx15k1(:,3*ks+3+levi(2,it),:),sp(:,k))
                        fijk=f3(is,js,ks,ijk)*di1*dj1*dk
                        call g1bval(d1b,sxzk,fijk)
                        call g2bval(d2b,sxzk,fijk)
                        call g3bval(d3b,sxzk,fijk)
                        call sxzupdate(sxzk,dk,sxzj2,k, &
                           sx15k2(:,3*ks+3+levi(1,it),:),sp(:,k))
                        fijk=-f3(is,js,ks,ijk)*di1*dj1*dk
                        call g1bval(d1b,sxzk,fijk)
                        call g2bval(d2b,sxzk,fijk)
                        call g3bval(d3b,sxzk,fijk)
                     enddo
                  enddo
               enddo
            enddo
         enddo
      enddo
   enddo
   end subroutine corpsi
\end{Verbatim}

\end{document}

