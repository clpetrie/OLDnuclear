\documentclass[12pt]{extarticle}
\usepackage[margin=1in]{geometry}
\usepackage{amssymb}
\usepackage{amsmath}
\usepackage{url}
\usepackage{bm}
\usepackage{color}

%My commands
\newcommand{\Oi}{\mathcal{O}_{i}}
\newcommand{\Oij}{\mathcal{O}_{ij}}
\newcommand{\Oijp}{\mathcal{O}^p_{ij}}
\newcommand{\Oklp}{\mathcal{O}^p_{kl}}
\newcommand{\ket}[1]{\left| #1 \right>}
\newcommand{\bra}[1]{\left< #1 \right|}
\newcommand{\braket}[2]{\left< #1 | #2 \right>}
\newcommand{\ketbra}[2]{\left| #1 \right> \left< #2 \right|}
\newcommand{\taui}{\bm{\tau}_i}
\newcommand{\tauj}{\bm{\tau}_j}
\newcommand{\sigmai}{\bm{\sigma}_i}
\newcommand{\sigmaj}{\bm{\sigma}_j}
\newcommand{\tauij}{\taui \cdot \tauj}
\newcommand{\sigmaij}{\sigmai \cdot \sigmaj}
\newcommand{\mycolor}[1]{\textcolor{red}{#1}}
\newcommand{\longsi}{s_1, \ldots, s_{i-1} , s, s_{i+1}, \ldots, s_A}
\newcommand{\longsij}{s_1, \ldots, s_{i-1} , s, s_{i+1}, \ldots, s_{j-1}, s', s_{j+1}, \ldots ,s_A}

\title{Calculating the Trial Wave Function for AFDMC}
\author{Cody Petrie}

\begin{document}
\maketitle

\section{Trial Wave Function}
The trial wave function for AFDMC must be simple to evaluate. In the past the simple Slater determinant with pair-wise correlations has been used as shown in \cite{gandolfi2014},
\begin{equation}
  \braket{RS}{\Psi_T} = \bra{RS} \left[ \prod_{i<j}f_c(r_{ij}) \right] \left[ 1+\sum_{i<j}\sum_p f_p(r_{ij})\Oijp \right] \ket{\Phi},
  \label{equ:simpletrial}
\end{equation}
where the $\Oijp$'s are $\tauij$, $\sigmaij$, and $t_{ij}\tauij$, where $t_{ij} = 3\sigmai \cdot \hat{r}_{ij} \sigmaj \cdot \hat{r}_{ij}-\sigmaij$. \textit{\mycolor{Why weren't $\sigmaij$ and $t_{ij}$ used in this paper?}}

My goal is to add the additional independent pair correlations.
\begin{equation}
  \braket{RS}{\Psi_T} = \bra{RS} \left[ \prod_{i<j}f_c(r_{ij}) \right] \left[ 1+\sum_{i<j}\sum_p f_p(r_{ij})\Oijp + \sum_{i<j}\sum_{k<l}\sum_p f_p(r_{ij})\Oijp f_p(r_{kl})\Oklp \right] \ket{\Phi},
\end{equation}

\section{Evaluation the Trial Wave Function}
To understand how to to this I'm going to just assume that $\Oijp$ only contains the term $\sigmaij$ and I'll start by looking at the trial wave function, equation~\ref{equ:simpletrial}, with only the linear term. So now
\begin{equation}
  \braket{RS}{\Psi_T} = \bra{RS} \left[ \prod_{i<j}f_c(r_{ij}) \right] \left[ 1+\sum_{i<j} f_1(r_{ij})\sigmaij \right] \ket{\Phi}.
\end{equation}
Also since the central correlations don't change the states by any more than a multiplicative factor I am going to ignore that term as well. I will also just look at one term in the sum (a particular $i$ and $j$ value). So we are just looking at
\begin{equation}
  \bra{RS} \left[ 1+f_1(r_{ij})\sigmaij \right] \ket{\Phi}.
  \label{equ:simpex}
\end{equation}
Now we also know that the Slater determinant is defined as
\begin{equation}
  \braket{RS}{\Phi} = \mathrm{det}(S) = \frac{1}{\sqrt{N!}} \begin{vmatrix}
  \phi_1(R_1S_1) & \phi_2(R_1S_1) & \cdots & \psi_N(R_1S_1) \\ 
  \phi_1(R_2S_2) & \phi_2(R_2S_2) & \cdots & \phi_N(R_2S_2) \\
  \vdots & \vdots & \ddots &\vdots \\
  \phi_1(R_NS_N)& \phi_2(R_NS_N) & \cdots & \phi_N(R_NS_N) \end{vmatrix},
\end{equation}
where $\phi_i(R_jS_j)=\phi^r_i(R_j)\phi^s_i(S_j)$ and $S$ is called the Slated Matrix.

Now lets look at equation~\ref{equ:simpex} again for an example.
\begin{align}
  & ~ ~ ~ \bra{RS} \left[ 1+f_1(r_{ij})\sigmaij \right] \ket{\Phi} \\
  &= \mathrm{det}(S) + f_1(r_{ij}) \bra{RS}\sigmaij\ket{\Phi} \\
  &= \mathrm{det}(S) + f_1(r_{ij})\mathrm{det}(S')
\end{align}
Here $S'$ is the updated matrix. It only has two columns different than $S$ and so we can get it's determinant of $S'$ easily once we have the determinant of $S$ by using the fact that
\begin{equation}
  \mathrm{det}(S^{-1}_{ij} S'_{jk}) = \frac{\mathrm{det}(S'_{jk})}{\mathrm{det}(S_{ij})}.
\end{equation}
When we solve for $\mathrm{det}(S)$ we finish solving for the inverse, $S^{-1}$ and the product $S^{-1}_{ij}S'_{jk}$ is $1$ on the diagonal and $0$ everywhere else except the two columns $i$ and $j$. This makes the $\mathrm{det}(S^{-1}_{ij}S'_{jk})$ easy to solve for since it is simply the determinant of the submatrix. Thus once we have $\mathrm{det}(S)$ it is easier to solve for $\mathrm{det}(S')$. All that is left is to do this over the pair loops and over each operator.

\section{Implimentation in the code}
Now how is this implimented into the code. The element of the Slater martix that corresponds to the $k^{th}$ orbital and the $i^{th}$ particle is given by
\begin{equation}
  S_{ki} = \braket{k}{r_i,s_i} = \sum_{s=1}^4 \braket{k}{r_i,s}\braket{s}{s_i}.
\end{equation}
From this you can see that a general Slater matrix can be written as a linear combination of matrix elements $\braket{k}{r_i,s}$ and coefficients $\braket{s}{s_i}$.

Therefore it's convenient to precompute
\begin{equation}
  \mathrm{sxz(s,i,j) = sxmallz(j,s,i)} = \sum_k S^{-1}_{jk} \braket{k}{r_i,s}.
\end{equation}
For example if we were computing the determinant of $S'_{ij} = \braket{k}{r_i,s'_i}$ where the $s'_i$ was changed is different from $s_i$ on the changed columns, then the product matrix could be computed as
\begin{equation}
  S^{-1}_{jk}S'_{ki} = \sum_{s=1}^4 \left(\sum_k S^{-1}_{jk} \braket{k}{r_i,s}\right) \left(\braket{s}{s_i}\right) = \sum_{s=1}^4 \mathrm{sxz(s,i,j)} \braket{s}{s_i}.
\end{equation}
\mycolor{Is this right?}

I have looked at how to calculate the trial wave function with a correlation operator in the middle now lets look at how to do it with 1 and 2-body spin-isospin operators in the middle. Here I am mostly filling in gaps in Kevin Schmidt's writeup.

\subsection{1-body spin-isospin operators}
Here the idea is we want to calculate expectation values like
\begin{equation}
 \left< \sum_i \Oi \right> = \frac{\bra{\Phi} \sum\limits_i \Oi \ket{R,S}}{\braket{\Phi}{R,S}}.
\end{equation}
Now let's expand this the numerator term
\begin{align}
  \bra{\Phi} \sum_i \Oi \ket{R,S} &= \bra{\Phi} \sum_i \Oi \ket{R,s_1,\ldots,s_A} \\
  &= \bra{\Phi} \sum_i\sum_{s=1}^4  \ketbra{s}{s} \Oi \ket{R,s_1,\ldots,s_A} \\
  &= \bra{\Phi} \sum_i\sum_{s=1}^4  \bra{s}\Oi\ket{s_i} \ket{R,\longsi} \\
  &= \bra{\Phi} \sum_i\sum_{s=1}^4 \alpha_{is} \ket{R,\longsi} \\
  &= \sum_i\sum_{s=1}^4 \alpha_{is} \braket{\Phi}{R,\longsi} \\
  &= \sum_i\sum_{s=1}^4 \alpha_{is} \mathrm{d1b(s,i)} \braket{\Phi}{RS},
\end{align}
where $\alpha_{is} = \bra{s} \Oi \ket{s_i}$ and $\mathrm{d1b(s,i)} = \braket{\Phi}{R,\longsi}/\braket{\Phi}{R,s_1,\ldots,s_A}$. Rearranging this we can get the expectation value
\begin{equation}
  \left< \sum_i \Oi \right> = \sum_i \sum_{s=1}^4 \alpha_{is}\mathrm{d1b(s,i)}.
\end{equation}
Notice that we have
\begin{equation}
  \mathrm{d1b(s,i) = sxz(s,i,i)}.
\end{equation}
\mycolor{I can't figure out why. Why is this? If it is easier to answer with regards to the d2b case then that works too.}

\subsection{2-body spin-isospin operators}
Here the idea is we want to calculate expectation values like
\begin{equation}
 \left< \sum_{i<j} \Oij \right> = \frac{\bra{\Phi} \sum\limits_{i<j} \Oij \ket{R,S}}{\braket{\Phi}{R,S}}.
\end{equation}
Now let's expand this the numerator term
\begin{align}
  \bra{\Phi} \sum_{i<j} \Oij \ket{R,S} &= \bra{\Phi} \sum_{i<j} \Oij \ket{R,s_1,\ldots,s_A} \\
  &= \bra{\Phi} \sum_{i<j}\sum_{s=1}^4\sum_{s'=1}^4  \ket{s}\braket{s}{s'}\bra{s'} \Oij \ket{R,s_1,\ldots,s_A} \\
  &= \bra{\Phi} \sum_{i<j}\sum_{s=1}^4\sum_{s'=1}^4  \bra{s,s'}\Oij\ket{s_i,s_j} \ket{R,\longsij} \\
  &= \bra{\Phi} \sum_{i<j}\sum_{s=1}^4\sum_{s'=1}^4 \alpha_{ijs} \ket{R,\longsij} \\
  &= \sum_{i<j}\sum_{s=1}^4\sum_{s'=1}^4 \alpha_{ijs} \braket{\Phi}{R,\longsij} \\
  &= \sum_{i<j}\sum_{s=1}^4\sum_{s'=1}^4 \alpha_{ijs} \mathrm{d2b(s,s',ij)} \braket{\Phi}{RS},
\end{align}
where $\alpha_{ijs} = \bra{s,s'} \Oij \ket{s_i,s_j}$ and
\begin{equation}
  \mathrm{d2b(s,s',ij)} = \braket{\Phi}{R,\longsij}/\braket{\Phi}{R,s_1,\ldots,s_A}.
\end{equation}
Rearranging this we can get the expectation value
\begin{equation}
  \left< \sum_{ij} \Oij \right> = \sum_{ij=1}^{((A-1)A)/2}\sum_{s=1}^4\sum_{s'=1}^4 \alpha_{ijs}\mathrm{d2b(s,s',ij)},
\end{equation}
and where the sum, $\sum\limits_{ij}^{((A-1)A)/2}$ is essentially doing the same thing as $\sum\limits_{i=1}^{A-1}\sum\limits_{j=i+1}^{A}$ or $\sum\limits_{i<j}$.
Notice again that we have
\begin{equation}
  \mathrm{d2b(s,s',ij)} = \mathrm{det} \begin{pmatrix}
  \mathrm{sxz(s,i,i)} & \mathrm{sxz(s,i,j)} \\
  \mathrm{sxz(s',j,i)} & \mathrm{sxz(s',j,j)} \end{pmatrix}
  =\mathrm{sxz(s,i,i)sxz(s',j,j) - sxz(s,i,j)sxz(s',j,i)}.
\end{equation}
\mycolor{I can't figure out why. Why is this? This doesn't need answered, I'll probably understand when I see the reason for the d1b case.}

\bibliographystyle{unsrt}
\bibliography{../../papers/references.bib}

\end{document}
